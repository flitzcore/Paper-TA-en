% Ubah judul dan label berikut sesuai dengan yang diinginkan.
\section{Conclusion}
\label{sec:kesimpulan}

% Ubah paragraf-paragraf pada bagian ini sesuai dengan yang diinginkan.

From the experiments conducted, it can be concluded that creation of chatbot using the Llama.cpp framework can be achieved and produces satisfying responses. This specifically with the aya-23-8B-Q4-K-M model, that has shown significant improvements in response quality when tested with 75 human-answered questions. The results of the tests showed that the model's reach Faithfulness of 93,095\%, Answer Relevancy of 98,299\%, and Answer Correctness of 77,093\% while only using 5,1 GB of VRAM and take about 6s to generate answer.