% Ubah judul dan label berikut sesuai dengan yang diinginkan.
\section{Related Research}
\label{sec:penelitianterkait}

% Ubah paragraf-paragraf pada bagian ini sesuai dengan yang diinginkan.
\subsection{Impact of chat bots usage in learning}
Research by Ms. Rona Riantini, which proposed simulation games as a learning medium, emphasizes the importance of effectively addressing problems on ships. Incorrect or slow handling can result in significant losses and even casualties. To train problem-solving skills, training for prospective crew members through electrical fault simulations on ships is necessary \cite{riantini2022serious}. This learning requires continuous practice and also needs adaptive feedback to effectively enhance student capabilities. Therefore, a teacher's presence is needed to guide students in understanding issues and directing problem-solving. However, the presence of teaching staff is less effective due to limitations in time and place. Many educators are needed to personally guide each student. Therefore, integrating a \emph{chatbot} as a guidance tool in the simulation of solving electrical problems on ships can offer significant benefits. With the \emph{chatbot}, consultation and guidance services can be provided to many students and can be conducted anytime. With the \emph{chatbot}, learning can be personalized for each student. Learning can adapt to the pace and learning style of each individual effectively and flexibly \cite{vazquezcano2021chatbot,kumar2021educational}. In research conducted by Essel, the performance of students at the University of Ghana trained by a \emph{chatbot} and those trained by an instructor was tested. From the results obtained, students interacting with the \emph{chatbot} showed better performance compared to the control group that interacted with a course instructor. Students trained by the \emph{chatbot} achieved an average score of 81.1, while those trained by an instructor scored 65.2. Students also felt very satisfied with the use of the chatbot, especially because it could provide instant feedback without delays in the interaction process \cite{essel2022virtual}.


\subsection{Related research on Retrieval Augmented Generation (RAG) method}
The \emph{Retrieval Augmented Generation} (RAG) technology enhances the answer generation capabilities of \emph{Large Language Models} (LLM) to be more accurate and relevant by integrating content retrieved from an external knowledge base to respond to user prompts \cite{xu2024nanjing}. This system is an advancement over previous systems that used \emph{Rule-Based} systems, meaning that users could only access the system by following a specific set of rules \cite{9501523}. This system was developed to address the rigid and limited shortcomings of the previous method. The \emph{chatbot} uses the LLM system to provide more natural answers, unlike instruction-based systems that are limited to specific instructions given by the user \cite{sarrouti2020sembionlqa}. RAG enables LLM to pull external data such as documents, datasets, and even internet access. This allows for taking a user's question, searching it from a data source, and then processing it into a natural response for the user \cite{tian2023intelligent}. To understand the provided data, an \emph{embeddings} process must be conducted on RAG to derive meaning from the data. To manage large documents, a \emph{chunking} method is used to divide the document into smaller parts to facilitate the information search process \cite{10448015}.


The system developed in this study includes several stages. First, words from external data sources and words from user prompts are converted into word vectors. For this, the concept of \emph{vector embedding} is used, which represents numerical vectors of data typically used to describe words, phrases, or even documents in a vector space \cite{kmetty2021presence}. In the concept of \emph{Neural Networks}, \emph{embedding} is a low-dimensional data representation that captures essential aspects of the original data. The concept of \emph{embedding} is widely used in language processing and recommendation systems. The main goal of \emph{embedding} is to transform non-numeric data, such as words or items, into numeric vectors so that computer \emph{neural network} systems can process them. One of the main advantages of \emph{embedding} is its ability to capture semantic relationships between data. In \emph{embedding} processing, words with similar contexts will have similar vector representations \cite{luo2018concept}.


Afterwards, the vectorized data is stored in a vector database with the assistance of the \emph{Facebook AI Similarity Search (FAISS)}, which creates vector indexes and uses an efficient and accurate nearest neighbor search algorithm for data retrieval \cite{10039758}. The FAISS algorithm utilizes the \emph{Product Quantization} (PQ) method, a technique for compressing high-dimensional data vectors into shorter, more efficient codes for storage and data retrieval. The article "A Survey of Product Quantization" explains that the PQ process involves dividing the input vector into smaller sub-vectors, each of which is then quantized independently. Each sub-vector is encoded into a specific identification from these independent parts, and these identifications are combined to form a PQ code that represents the original vector. One of the main advantages of PQ is its ability to efficiently estimate the distance between the original vector and its PQ code using \emph{Asymmetric Distance Computation} (ADC). This method allows for fast searching within very large databases using a \emph{lookup table}. Subsequently, the data retrieval results and user prompts are input into the LLM model to generate a more accurate final answer. The RAG method integrates the retrieval model and LLM, extracts semantic features from questions and answers, and ranks results based on semantic relevance between texts, then inputs the retrieval results into a large language model for enhanced generation, thus achieving a knowledge-based smart question answering system.





% Beberapa penelitian lain pernah dilakukan seperti yang dirumuskan oleh \citet{newton1687} bahwa \lipsum[5]
% Hasil tersebut kemudian menjadi persamaan \ref{eq:hukumpertama}.

% % Contoh pembuatan persamaan ilmiah.
% \begin{equation}
%   \label{eq:hukumpertama}
%   \sum \mathbf{F} = 0\; \Leftrightarrow\; \frac{\mathrm{d} \mathbf{v} }{\mathrm{d}t} = 0.
% \end{equation}

