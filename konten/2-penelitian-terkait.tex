% Ubah judul dan label berikut sesuai dengan yang diinginkan.
\section{Penelitian Terkait}
\label{sec:penelitianterkait}

% Ubah paragraf-paragraf pada bagian ini sesuai dengan yang diinginkan.
\subsection{Dampak Penggunaan Chat Bot dalam Pembelajaran}
Penelitian oleh Ibu Rona Riantini yang mengusulkan permainan simulasi sebagai media pembelajaran, menjelaskan mengenai pentingnya mengatasi permasalahan pada kapal dengan efektif. Penanganan yang salah atau lambat dapat mengakibatkan kerugian yang besar bahkan korban jiwa. Untuk melatih kemampuan pemecahan masalah ini diperlukan pelatihan bagi calon awak kapal melalui simulasi kerusakan listrik pada kapal \cite{riantini2022serious}. Pembelajaran ini membutuhkan latihan yang berkelanjutan dan juga membutuhkan umpan balik yang adaptif untuk meningkatkan kemampuan pelajar secara  efektif. Oleh karena itu perlu sosok guru untuk membimbing pelajar dalam memahami permasalahan dan mengarahkan dalam pemecahan masalah. Namun, sosok tenaga pengajar kurang efektif karena terbatas pada tempat dan waktu. Perlu banyak tenaga pengajar untuk membimbing setiap pelajar secara pribadi. Oleh karena itu, mengintegrasikan \emph{chat bot} sebagai alat bimbingan dalam simulasi pemecahan masalah kelistrikan di kapal dapat memberi keuntungan yang signifikan. Dengan adanya \emph{chat bot}, dapat diberikan layanan konsultasi dan bimbingan bagi banyak mahasiswa dan bisa dilakukan kapan saja. Dengan adanya \emph{chat bot} dapat diberikan pembelajaran yang menyesuaikan pada setiap pelajar secara pribadi. Pembelajaran dapat menyesuaikan dengan kecepatan dan gaya belajar setiap orang secara efektif dan fleksibel \cite{vazquezcano2021chatbot,kumar2021educational}. Dalam penelitian yang dilakukan oleh Essel, diuji kinerja mahasiswa pada Universitas Ghana yang dilatih oleh \emph{chat bot} dan mereka yang dilatih  oleh instruktor. Dari hasil yang diperoleh,  mahasiswa yang berinteraksi dengan \emph{chat bot} menunjukkan performa lebih baik dibandingkan dengan kelompok kontrol yang berinteraksi dengan instruktur kursus. Mahasiswa yang dilatih oleh \emph{chat bot} memperoleh nilai rata-rata 81,1 sedangkan mereka yang dilatih oleh instruktor memperoleh nilai 65,2. Mahasiswa juga merasa sangat puas dengan penggunaan chatbot, terutama karena dapat memberikan umpan balik instan tanpa mengalami keterlambatan dalam proses interaksi \cite{essel2022virtual}.

\subsection{Penelitian Terkait Metode Retrieval Augmented Generation (RAG)}
Teknologi \emph{Retrieval Augmented Generation }(RAG) meningkatkan kemampuan generasi jawaban \emph{Large Language Model} (LLM) menjadi lebih akurat dan relevan dengan mengintegrasikan konten yang diambil dari basis pengetahuan eksternal untuk menjawab prompt dari pengguna \cite{xu2024nanjing}. Sistem ini merupakan kemajuan dari sistem sebelumnya yang menggunakan sistem \emph{Rule Based} yang berarti pengguna hanya bisa mengakses sistem dengan mengikuti seperangkat aturan tertentu \cite{9501523}. Sistem ini dibuat untuk mengatasi kelemahan cara sebelumnya yang kaku dan terbatas. \emph{Chat bot} menggunakan sistem (LLM) untuk menyediakan jawaban dengan lebih natural berbeda dengan sistem berbasis instruksi yang terbatas pada instruksi tertentu yang diberikan pengguna \cite{sarrouti2020sembionlqa}. RAG memungkinkan LLM untuk mengambil data eksternal berupa dokumen, dataset, bahkan akses internet. Hal ini memungkikan untuk mengambil pertanyaan pengguna, mencarinya dari sumber data lalu mengolah menjadi jawaban yang natural bagi pengguna \cite{tian2023intelligent}. Untuk dapat mengerti data yang diberikan, harus dilakukan proses \emph{embeddings} pada RAG untuk memperoleh makna dari data. Untuk mengatasi dokumen besar, dilakukan metode \emph{chunking} untuk mebagi dokumen menjadi lebih kecil agar mempermudah proses pencarian informasi \cite{10448015}. 

Sistem yang dibuat pada penelitian ini meliputi beberapa tahapan. Pertama, kata-kata pada sumber data eksternal dan kata-kata pada prompt pengguna diubah menjadi vektor kata. Untuk itu digunakan konsep \emph{vector embedding} yang merepresentasi vektor numerik dari data yang biasanya digunakan untuk menggambarkan kata-kata, frasa, atau bahkan dokumen dalam ruang vektor \cite{kmetty2021presence}. Dalam konsep \emph{Neural Network}, \emph{embedding}adalah representasi data berdimensi rendah yang menangkap aspek-aspek penting dari data asli. Konsep \emph{embedding} digunakan secara luas dalam pemrosesan bahasa dan sistem rekomendasi. Tujuan utama dari \emph{embedding} adalah untuk mengubah data non-numerik, seperti kata-kata atau item, menjadi vektor numerik sehingga sistem \emph{neural network} komputer dapat memprosesnya. Salah satu keuntungan utama dari \emph{embedding} adalah kemampuannya untuk menangkap hubungan semantik antar data. Dalam pemrosesan \emph{embedding}, kata-kata dengan konteks yang serupa akan memiliki representasi vektor yang mirip \cite{luo2018concept}.

Setelah itu, data yang telah di-vektorisasi disimpan dalam basis data vektor dengan bantuan  \emph{Facebook AI Similarity Search (FAISS)}, yang menciptakan indeks vektor dan menggunakan algoritma pencarian tetangga terdekat yang efisien dan akurat untuk pengambilan data \cite{10039758}. Algoritma FAISS memanfaatkan metode \emph{Product Quantization} (PQ) yang merupakan teknik untuk mengompresi vektor data berdimensi tinggi menjadi kode pendek yang lebih efisien untuk penyimpanan dan pencarian data. Dalam artikel "A Survey of Product Quantization" dijelaskan bahawa proses PQ melibatkan pembagian vektor input menjadi sub-vektor yang lebih kecil, yang masing-masing kemudian dikuantisasi secara independen. Setiap sub-vektor dikodekan ke dalam identifikasi tertentu dari bagian independen ini, dan identifikasi-identifikasi ini digabungkan untuk membentuk sebuah kode PQ yang mewakili vektor asli. Salah satu keuntungan utama dari PQ adalah kemampuannya untuk memperkirakan jarak antara vektor asli dan kode PQ-nya dengan cara yang efisien dengan perhitungan \emph{Asymmetric Distance Computation}, (ADC). Metode ini memungkinkan pencarian cepat dalam database yang sangat besar dengan menggunakan \emph{lookup table}. Selanjutnya, hasil pengambilan data dan prompt pengguna dimasukkan ke dalam model LLM untuk menghasilkan jawaban akhir yang lebih akurat. Metode RAG mengintegrasikan model pengambilan dan LLM, mengekstrak fitur semantik dari pertanyaan dan jawaban serta mengingat hasil berdasarkan relevansi semantik antarteks, kemudian memasukkan hasil pengambilan ke dalam model bahasa besar untuk generasi yang ditingkatkan, sehingga mencapai sistem penjawab pertanyaan cerdas berbasis pengetahuan. 




% Beberapa penelitian lain pernah dilakukan seperti yang dirumuskan oleh \citet{newton1687} bahwa \lipsum[5]
% Hasil tersebut kemudian menjadi persamaan \ref{eq:hukumpertama}.

% % Contoh pembuatan persamaan ilmiah.
% \begin{equation}
%   \label{eq:hukumpertama}
%   \sum \mathbf{F} = 0\; \Leftrightarrow\; \frac{\mathrm{d} \mathbf{v} }{\mathrm{d}t} = 0.
% \end{equation}

