% Ubah judul dan label berikut sesuai dengan yang diinginkan.
\section{Introduction}
\label{sec:pendahuluan}

% Ubah paragraf-paragraf pada bagian ini sesuai dengan yang diinginkan.

Solving electrical problems on ships remains a process that requires a lot of practice and technical experience. To provide this technical experience, direct training on ships is necessary. However, not all institutions can provide the ships needed to train students. To aid this training process, research by Ms. Rona Riantini has proposed simulation games as a learning medium \cite{riantini2022serious}. To provide more effective training, it is suggested to use a chatbot to assist participants in registration and use of the simulation. Additionally, the chatbot is also designed to help participants understand Earth Fault in detail. The chatbot operates using a website medium. A chatbot is a system used to respond to user questions using their language. The use of a chatbot will be implemented with Natural Language Processing (NLP) to produce better language recognition. NLP is a branch of artificial intelligence that focuses on understanding and generating human text or language by computers. Chatbots using NLP have the ability to understand, process, and respond to human text or speech automatically. NLP itself has limitations on the limited information it possesses. NLP can only access information provided during the training process. Therefore, a Retrieval Augmented Generation (RAG) system is needed. RAG allows LLM to retrieve external data in the form of documents, datasets, even internet access. To understand the data provided, the RAG process involves embeddings to derive meaning from the data. To handle large documents, a chunking method is employed to divide the document into smaller parts to facilitate the information search process. After obtaining relevant information, NLP processes this information into answers to user questions. The system will be created in the form of a website and will be connected to the earth fault simulation website. The system is designed to provide natural interaction to users. This system is expected to serve as a question-and-answer facility to help users in using the simulation system, and also to help users understand the concept of Earth Fault in depth.

The discussion in this paper begins with a presentation on other research (\ref{sec:penelitianterkait}).
It is then followed by an explanation of the architecture of the system created (\ref{sec:arsitektur}).
Based on this, we show the results obtained from the system created (\ref{sec:analisahasil}).
Finally, conclusions are drawn from the research that has been conducted (\ref{sec:kesimpulan}).

