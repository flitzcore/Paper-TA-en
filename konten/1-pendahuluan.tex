% Ubah judul dan label berikut sesuai dengan yang diinginkan.
\section{Pendahuluan}
\label{sec:pendahuluan}

% Ubah paragraf-paragraf pada bagian ini sesuai dengan yang diinginkan.

Pemecahan masalah kelistrikan kapal tetap merupakan proses yang membutuhkan banyak latihan dan pengalaman teknis. Untuk memberi pengalaman teknis ini, perlu dilakukan pelatihan sercara langsung diatas kapal. Namun, tidak semua institusi mampu menyediakan kapal yang diperlukan untuk melatih pelajar
. Untuk membantu proses pelatihan ini, telah dilakukan penelitian oleh Ibu Rona Riantini yang mengusulkan permainan simulasi sebagai media pembelajaran \cite{riantini2022serious}. Untuk menyediakan pelatihan yang lebih efektif, diusulkan penggunaan \emph{chat bot} untuk membantu peserta dalam registrasi dan penggunaan simulasi. Selain itu, \emph{chat bot} juga  dirancang untuk membantu peserta memahami \emph{Earth Fault} secara detail. \emph{Chat bot} dengan menggunakan media website dalam operasinya. \emph{Chat bot} adalah sistem yang digunakan untuk merespon pertanyaan yang diberikan pengguna dengan dengan menggunakan bahasa mereka \cite{9501523}. Penggunaan \emph{chat bot} akan diimplementasikan dengan \emph{Natural Language Processing} (NLP),  untuk menghasilkan pengenalan bahasa yang lebih bagus. NLP adalah cabang dari kecerdasan buatan yang fokus pada pemahaman dan penghasilan teks atau bahasa manusia oleh komputer. \emph{Chat bot} yang menggunakan NLP memiliki kemampuan untuk memahami, memproses, dan merespons teks atau ucapan manusia secara otomatis \cite{9402401}. NLP sendiri memiliki kelemahan pada terbatasnya informasi yang dimiliki. NLP hanya dapat mengakses informasi yang diberikan pada saat proses pelatihan. Oleh karena itu, diperlukan sistem \emph{Retrieval Augmented Generation} (RAG). RAG memungkinkan LLM untuk mengambil data eksternal berupa dokumen, dataset, bahkan akses internet. Untuk dapat mengerti data yang diberikan, harus dilakukan proses \emph{embeddings} pada RAG untuk memperoleh makna dari data. Untuk mengatasi dokumen besar, dilakukan metode \emph{chunking} untuk mebagi dokumen menjadi lebih kecil agar mempermudah proses pencarian informasi. Setelah memperoleh informasi yang relevan, informasi akan diolah oleh NLP menjadi jawaban untuk pertanyaan pengguna \cite{10448015}. Sistem akan dibuat dalam bentuk website, dan akan dihubungkan dengan website simulasi \emph{earth fault}. Sistem dirancang untuk memberi interaksi natural pada pengguna. Sistem ini diharapkan bisa menjadi sarana tanya jawab untuk membantu pengguna dalam menggunakan sistem simulasi, dan juga untuk membantu pengguna memahami konsep \emph{earth fault} secara mendalam. 

Pembahasan pada paper ini dimulai dengan presentasi mengenai penelitian lain (Bagian \ref{sec:penelitianterkait}).
Kemudian dilanjutkan dengan penjelasan mengenai arsitektur dari sistem yang dibuat (Bagian \ref{sec:arsitektur}).
Berdasarkan hal tersebut, kami menunjukkan hasil yang diperoleh dari sistem yang dibuat(Bagian \ref{sec:loremipsum}).
Terakhir, didapatkan kesimpulan dari penelitian yang telah dilakukan (Bagian \ref{sec:kesimpulan}).
