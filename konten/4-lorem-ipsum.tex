% Ubah judul dan label berikut sesuai dengan yang diinginkan.
\section{Analisa Hasil}
\label{sec:analisahasil}

Selanjutnya dilakukan analisa terhadap hasil yang diperoleh dari model LLM. Hasil yang diperoleh akan dianalisa berdasarkan relevansi dan akurasi jawaban yang diberikan. 

\subsection{Hasil tes framework RASA}
Yang pertama adalah pengujian dengan menggunakan \emph{framework} RASA. Dari pengujian ini, ditemukan bahwa RASA memiliki kelebihan dalam hal \emph{deployment} yang mudah. Namun, RASA memiliki kekurangan dalam hal \emph{training}. RASA melakukan \emph{training} dengan menggunakan data yang sudah ada. RASA menggunakan file YAML untuk menyimpan konfigurasinya. Konfigurasi ini harus berisi data-data yang ingin digunakan. Konverasi akan terbatas pada data yang sudah ada ini. Hal ini menyebabkan RASA tidak dapat merespon pertanyaan diluar data ini. Jika pengguna memberi input yang berbeda dari data, RASA akan kesulitan untuk mengerti niat pengguna. Hal ini menyebabkan RASA tidak dapat memberikan jawaban yang sesuai dengan pertanyaan pengguna. RASA dapat digunakan dengan efektif jika pengguna memiliki dataset tanya jawab yang luas sehingga RASA dapat memberikan jawaban yang sesuai. RASA juga tidak memiliki komunitas yang aktif sehingga perkembangan RASA lebih lambat ketimbang \emph{framework} lain. Hal ini menyebabkan RASA kurang optimal untuk digunakan dalam proyek ini.

\subsection{Framework Langchain dan HuggingFace}
Selanjutnya adalah pengujian dengan menggunakan gabungan LangChain dan HuggingFace. Dari hasil pengujian diperoleh hasil pada Tabel \ref{tab:hasilrag}
\begin{table}[!htbp]
    \caption{Hasil Framework Langchain dan HuggingFace}
    \label{tab:hasilrag}
    \centering
    \begin{tabular}{llll}
      \toprule
      Nama                      & RAM(GB)  & VRAM(GB)  & Analisa  \\
      \midrule
      bigscience/bloom-7b1      & 2.5      & 8.2       & Hasil tepat, tapi\\ \cite{muennighoff2022crosslingual}                          &          &           & kurang natural \\  
      \\
      bigscience/bloom-1b7
      \cite{muennighoff2022crosslingual}      & 3.1      & 6.9       & Hasil tepat, tapi\\
                                &          &           & kurang natural \\ 
      \\ 
      sail/Sailor-4B \cite{dou2024sailor}           & 4.3      & 6.3       & Hasil tidak sesuai\\
      \\
      sail/Sailor-0.5B \cite{dou2024sailor}          & 2.3      & 2.7       & Hasil tidak sesuai\\
      \\
      indonlp/cendol-mt5-       & 3.0      & 0.7       & Hasil tepat, tapi\\
      small-inst \cite{indonlp1,indonlp2, indonlp3, indonlp4, indonlp5, indonlp6, indonlp7}              &          &           & kurang natural \\ 
      \\
      indonlp/cendol-llama2-    & 2.5    & 0.7         & Hasil tidak sesuai\\
      7b-chat \cite{indonlp1,indonlp2, indonlp3, indonlp4, indonlp5, indonlp6, indonlp7}                  &          &           &  \\
      \\
      Yellow-AI-NLP/komodo-     & 5.0      & 7.3       & Hasil tidak sesuai\\
      7b-base \cite{owen2024komodo}                  &          &           &  \\
      \\
      cahya/gpt2-small-         & 2.4      & 0.4       & Hasil tidak sesuai\\
      indonesian-522M \cite{cahya_llm}          &          &           &  \\
      \\
      kalisai/Nusantara-7b-     & 3.2      & 9.9       & Hasil tepat, tapi\\
      Indo-Chat \cite{zulfikar_aji_kusworo_2024}        &          &           & kurang natural \\
      \\
      kalisai/Nusantara-4b-     & 4.4      & 6.3       & Hasil tepat, tapi\\
      Indo-Chat  \cite{zulfikar_aji_kusworo_2024}               &          &           & kurang natural \\
      \bottomrule
    \end{tabular}
  \end{table}

\subsection{Framework Langchain dan Llama.cpp}
Selanjutnya adalah pengujian dengan menggunakan gabungan LangChain dan Llama.cpp. Dari hasil pengujian diperoleh hasil pada Tabel \ref{tab:hasilllama}
\begin{table}[!htbp]
    \caption{Hasil llama.cpp}
    \label{tab:hasilllama}
    \centering
    \begin{tabular}{llll}
    \toprule
    Nama                      & RAM(GB)  & VRAM(GB)  & Analisa  \\
    \midrule
    bigscience/bloom-7b1 \cite{muennighoff2022crosslingual}      & 2.2      & 7.2       & Hasil tepat, tapi\\
                              &          &           & kurang natural \\  
    \\
    bigscience/bloom-1b7 \cite{muennighoff2022crosslingual}      & 3.1      & 5.0       & Hasil tepat, tapi\\
                              &          &           & kurang natural \\ 
    \\ 
    Merak-7B-v4-model-Q4 \cite{Merak}     & 3.4      & 5.3       & Hasil tepat, tapi\\
                              &          &           & kurang natural \\  
    \\
    Merak-7B-v4-model-Q5 \cite{Merak}     & 3.5      & 6.0       & Hasil tepat, tapi\\
                              &          &           & kurang natural \\
    \bottomrule
  \end{tabular}
\end{table}
% Ubah paragraf-paragraf pada bagian ini sesuai dengan yang diinginkan.

% % Contoh input beberapa gambar pada halaman.
% \begin{figure*}
%   \centering
%   \subfloat[Hasil A]{\includegraphics[width=.4\textwidth]{example-image-a}
%     \label{fig:hasila}}
%   \hfil
%   \subfloat[Hasil B]{\includegraphics[width=.4\textwidth]{example-image-b}
%     \label{fig:hasilb}}
%   \caption{Contoh input beberapa gambar.}
%   \label{fig:hasil}
% \end{figure*}

% \lipsum[16-18]

% % Contoh input potongan kode dari file.
% \lstinputlisting[
%   language=Python,
%   caption={Program perhitungan bilangan prima.},
%   label={lst:bilanganprima}
% ]{program/bilangan-prima.py}

% \lipsum[19-20]
