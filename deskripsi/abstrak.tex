% Mengubah keterangan `Abstract` ke bahasa indonesia.
% Hapus bagian ini untuk mengembalikan ke format awal.
\renewcommand\abstractname{Abstrak}

\begin{abstract}

  % Ubah paragraf berikut sesuai dengan abstrak dari penelitian.
  Pemecahan masalah kelistrikan kapal tetap merupakan proses yang menantang dan membutuhkan banyak latihan, sehingga diperlukan media pembelajaran simulasi yang melatih pelajar untuk mengatasi masalah kelistrikan. Pelajar tentunya membutuhkan bimbingan selama pelatihan untuk memperoleh hasil yang maksimal. Bimbingan konvensional berupa diskusi dengan pembimbing, tentunya tidak efektif karena akan sangat dibatasi oleh waktu dan sumber daya. Diusulkan penggunaan \emph{chat bot} untuk menyediakan media bertanya dan konsultasi efektif untuk menyelesaikan kendala kelistrikan pada kapal, dengan menggunakan \emph{ Natural Language Understanding} (NLU), untuk
  dapat mengerti intensi dari pertanyaan secara tepat. \emph{Chat bot} dirancang untuk membantu peserta dalam registrasi dan penggunaan simulasi. Selain itu, chat bot juga dirancang untuk membantu peserta memahami \emph{Earth Fault} secara detail. Dengan adanya \emph{chat bot}, peserta pelatihan dapat berkonsultasi tanpa terbatas oleh waktu kerja dan zona waktu.
\end{abstract}

% Mengubah keterangan `Index terms` ke bahasa indonesia.
% Hapus bagian ini untuk mengembalikan ke format awal.
\renewcommand\IEEEkeywordsname{Kata kunci}

\begin{IEEEkeywords}

  % Ubah kata-kata berikut sesuai dengan kata kunci dari penelitian.
  Natural Language Understanding, Chat Bot, Earth Fault, Simulasi.

\end{IEEEkeywords}
