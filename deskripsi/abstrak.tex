% Mengubah keterangan `Abstract` ke bahasa indonesia.
% Hapus bagian ini untuk mengembalikan ke format awal.
\renewcommand\abstractname{Abstract}

\begin{abstract}

  % Ubah paragraf berikut sesuai dengan abstrak dari penelitian.
  Troubleshooting of ship electrical problems remains a challenging process and requires a lot of practice. There is a need for simulated learning media that trains students to solve electrical problems. Students of course need guidance during training to obtain maximum results. Conventional guidance in the form of discussions with the mentor is certainly not effective because it is limited by time and resources. It is proposed to use chatbots to provide an effective medium of questioning and consultation for solving electrical problems on ships, by using Natural Language Understanding (NLU) to precisely understand the intention of the question. The chatbot is designed to assist participants in registration and use of the simulation. In addition, the chatbot was also designed to help participants understand Earth Fault in detail. With the chatbot, trainees can consult without being limited by working time and time zone.

\end{abstract}

% Mengubah keterangan `Index terms` ke bahasa indonesia.
% Hapus bagian ini untuk mengembalikan ke format awal.
\renewcommand\IEEEkeywordsname{Keywords}

\begin{IEEEkeywords}

  % Ubah kata-kata berikut sesuai dengan kata kunci dari penelitian.
  Natural Language Understanding, Chat Bot, Earth Fault, Simulation.

\end{IEEEkeywords}
